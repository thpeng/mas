%! Date = 28/03/2023

% Preamble
\documentclass[a4paper,12pt]{article}

\usepackage[utf8]{inputenc}
\usepackage[T1]{fontenc}
\usepackage{lmodern}
\usepackage[english]{babel}
\usepackage{biblatex}
\addbibresource{references.bib}

\begin{document}
    \begin{titlepage}
        \begin{center}
        {\huge\bfseries Eignung von einem Internal Developer Portal zur Unterstützung der täglichen Arbeit von BizDevOps Engineers\par}
            \vspace{2cm}

            {\scshape\large Zertifikatsarbeit im \par}
            {\scshape\large CAS Digital Product Lead \par}
            \vspace{1cm}

            {\scshape\large Zürcher Fachhochschule \par}
            {\scshape\large HWZ Hochschule für Wirtschaft Zürich \par}
            \vspace{4cm}

            {\normalsize eingereicht bei:\par}
            \vspace{0.5cm}

            {\large Ralph Hutter\par}
            \vfill
            {\normalsize vorgelegt von:\par}
            \vspace{0.5cm}
            {\large Thierry Peng\par}
            \vspace{0.5cm}
            {\normalsize Adresse: placeholder 68, placeholder\par}
            {\normalsize  Ort, Datum: placeholder, \today\par}

        \end{center}
    \end{titlepage}


    \section*{Management Summary}

    test.one-two-three
    \cite{backstageio}
    \pagebreak


    \tableofcontents
    \pagebreak


    \section*{Ehrenwörtliche Erklärung}

    Ich bestätige hiermit, dass ich
    \begin{itemize}
        \item die vorliegende Thesis selbstständig und ohne Benützung anderer als der angegebenen Quellen oder Hilfsmittel anfertigte,
        \item die benutzten Quellen wörtlich oder inhaltlich als solche kenntlich machte,
        \item diese Arbeit in gleicher oder ähnlicher Form noch keiner Prüfungskommission vorlegte.
    \end{itemize}
    placeholder, \today\newline


    \section{Einleitung}


    \section{Inhaltlicher Teil}

    \subsection{Definition}
    At the core, an Internal Developer Portal (Portal) is a catalogue of information, which supports the daily work of
    engineers
    operating IT systems, developing software or other related work.
    By itself, the portal is not the single point of truth of the information, but usually aggregates them from several sources.
    In context of a portal, several sources mention another concept - a internal developer platform.
    It's worth exploring this concept to discover the value of
    a portal.

    \subsubsection{Internal Developer Platform}
    Gartner does differentiate the concept of a Internal Developer Portal and a Internal Developer Platform with the
    following quote: 'internal developer portals servce as the interface through wicht developers can discover and
    access internal developer platform capabilities' \cite{gartner}.

    A Internal Developer Platform (Platform), according to the COO Christoph C. Richter from Humanitec\cite{richteretal},
    should consist at least of the following component:
    \begin{itemize}
        \item Application Configuration Management
        \item Infrastructure Orchestration
        \item Environment Management
        \item Deployment Management
        \item Role-Based Access Control
    \end{itemize}
    In some other sources, there are also additional components mentioned, e.g. Observablity\cite{xenon}. XENONSTACK as
    well as Humanitec mentioned before, are suppliers of Internal Developer Platform solutions with different capabilities.
    The capabilities and components of Internal Developer Platforms are not unique to the products of those two suppliers.
    In most contemporary IT operations or software development departments these components and practices are well-known
    and used widely.
    Beyond the size of a very small team, it is necessary to use tooling to know where, in which version and with what
    configuration a software artifact was installed.
    The reason behind this may be to coordinate a new software rollout, test the software on a dedicated stage, fix bugs
    or develop new features and to solve incidents caused by your software.
    Thus, it can be argued, that the concept of an Internal Developer Platform is not unique to specific products
    or offerings, but can be used on a set of tooling and practices supporting IT operations and software development.
    This set of tooling and practices can be viewed as predecessors of Internal Developer Platforms and are usually
    distinct solutions or products.
    For example there is a seperate Configuration Management Datatabase
    to address the need of Application Configuration Management or there is a purpose built Continuous Integration /
    Continuous Deployment pipeline for addressing the need for a Deployment Management.
    These solutions are not necessarily integrated with each other.
    It may be also the case, that these
    products or solutions were established before the advent of Agile Methodologies and DevOps.

    \subsubsection{DevOps}
    With DevOps, the classical disciplines of IT operating and software development are much more integrated to reduce
    friction, release features more often and to improve quality of the products\cite{safedevops}.
    Before DevOps it was quite common to have different toolings for either development or operating roles.
    One of the fundamental principles of DevOps was coined by the current CTO of Amazon, Werner Vogels\cite{vogels};
    'you build it, you run it'.
    This means, that the same team develops the software and operates it.
    With DevOps, a team takes the full responsibility over the lifecycle for a software or a platform.
    In combination with agile methodologies such as Scrum or SAFe even business oriented people are intergrated in this
    cross-functional team.
    One example could be the Product Owner\cite{safepo}.
    This has also the implication, that a team must have a wide range of skills and should have access to different
    tools needed.
    Such tools can be needed for activities such as requirements engineering, solution architecting, software development
    but also everything that is necessary to operate the software.
    Especially the operating of the software needs various tools. There are tools for logging and monitoring, the
    configuration management database to find out whom belongs a particular application or platform but also incident
    handling and alarming system.
    While DevOps has various benefits, it adds quite a complexity for a team member in a agile and DevOps team.
    That's where the Intergrated Developer Portal has various proposals to support the individual and the team in their
    daily business.

    \subsubsection{Internal Developer Platform}


    \subsection{Systematische Betrachtung}


    \section{Schlussfolgerung}


    \section{Quellenverzeichnis}

    \printbibliography[heading=none]


\end{document}