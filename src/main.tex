%! Date = 28/03/2023

% Preamble
\documentclass[a4paper,12pt]{article}

\usepackage[utf8]{inputenc}
\usepackage[T1]{fontenc}
\usepackage{lmodern}
\usepackage[ngerman]{babel}
\usepackage{biblatex}
\addbibresource{references.bib}

\begin{document}
    \begin{titlepage}
        \begin{center}
        {\huge\bfseries Eignung von einem Internal Developer Portal zur Unterstützung der täglichen Arbeit von BizDevOps Engineers\par}
            \vspace{2cm}

            {\scshape\large Zertifikatsarbeit im \par}
            {\scshape\large CAS Digital Product Lead \par}
            \vspace{1cm}

            {\scshape\large Zürcher Fachhochschule \par}
            {\scshape\large HWZ Hochschule für Wirtschaft Zürich \par}
            \vspace{4cm}

            {\normalsize eingereicht bei:\par}
            \vspace{0.5cm}

            {\large Ralph Hutter\par}
            \vfill
            {\normalsize vorgelegt von:\par}
            \vspace{0.5cm}
            {\large Thierry Peng\par}
            \vspace{0.5cm}
            {\normalsize Adresse: placeholder 68, placeholder\par}
            {\normalsize  Ort, Datum: placeholder, \today\par}

        \end{center}
    \end{titlepage}


    \section*{Management Summary}

    test.one-two-three
    \cite{backstageio}
    \pagebreak


    \tableofcontents
    \pagebreak


    \section*{Ehrenwörtliche Erklärung}

    Ich bestätige hiermit, dass ich
    \begin{itemize}
        \item die vorliegende Thesis selbstständig und ohne Benützung anderer als der angegebenen Quellen oder Hilfsmittel anfertigte,
        \item die benutzten Quellen wörtlich oder inhaltlich als solche kenntlich machte,
        \item diese Arbeit in gleicher oder ähnlicher Form noch keiner Prüfungskommission vorlegte.
    \end{itemize}
    placeholder, \today\newline

    \section{Einleitung}
    \section{Inhaltlicher Teil}
    \subsection{was ist eine Internal Developer Platform}

    \subsection{Was ist ein Internal Developer Portal}
    Ein Internal Developer Portal ist ein Teil von einer 'Internal Developer Platform' \cite{idporgdevportal}. Das
    Portal fungiert gemäss Gartner\cite{gartner} als Oberfläche für die Plattform.



    \subsection{Systematische Betrachtung}

    \section{Schlussfolgerung}
    \section{Quellenverzeichnis}

    \printbibliography[heading=none]


\end{document}