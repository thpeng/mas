%! Date = 28/03/2023

% Preamble
\documentclass[a4paper,10pt]{article}

\usepackage[utf8]{inputenc}
\usepackage[T1]{fontenc}
\usepackage{lmodern}
\usepackage[english]{babel}
\usepackage{graphicx}
\usepackage[style=authoryear]{biblatex}
\usepackage{url}
\usepackage{hyperref}
\usepackage{booktabs}
\usepackage{tabularx}
\usepackage{pdflscape}
\usepackage{placeins}
\usepackage{geometry}
\usepackage{textcomp}

\geometry{
    a4paper,
    total={170mm,257mm},
    left=30mm,
    right=30mm,
    top=25mm,
    bottom=25mm,
}
\linespread{1.5}
\addbibresource{references.bib}

\begin{document}
    \begin{titlepage}
        \begin{center}
        {\huge\bfseries Suitability of an Internal Developer Portal to Support the Daily Work of BizDevOps Engineers\par}
            \vspace{2cm}

            {\scshape\large Certificate work in the \par}
            {\scshape\large CAS Digital Product Lead \par}
            \vspace{1cm}

            {\scshape\large HWZ University of Applied Sciences \par}
            {\scshape\large for Business Administration Zurich \par}
            \vspace{4cm}

            {\normalsize submitted to\par}
            \vspace{0.5cm}

            {\large Ralph Hutter\par}
            \vfill
            {\normalsize submitted by\par}
            \vspace{0.5cm}
            {\large Thierry Peng\par}
            \vspace{0.5cm}
            {\normalsize Address: placeholder 68, placeholder\par}
            {\normalsize  Place, Date: placeholder, \today\par}

        \end{center}
    \end{titlepage}


    \section*{Management Summary}
    An Internal Developer Portal is a software that aims to provide a centralized and unified view to support the daily
    work of DevOps engineers.
    Internal developer portals have been developed by several companies to reduce the so-called ``cognitive load`` of
    their DevOps engineers.
    At its core, it serves as a catalog that includes elements such as teams, users, applications, platforms, and resources.
    It visualizes the relationships between these items and gives users an overview of the immediate context in
    which they are working.
    Additionally, various use cases have been developed based on the implementations of an Internal Developer Portal.
    These include features that assist DevOps teams in managing cloud costs, blueprints for building applications from
    the ground up, and tools for organizing technical documentation and facilitating internal communication.\\ \\
    SBB IT adopted cloud technologies in 2016 and has since built a comprehensive, agile DevOps organization for
    developing and running its business applications in the cloud.
    In some cases, SBB IT still uses development tools from its previous organization.
    While these tools meet the requirements to be called an ``Internal Developer Platform``, mostly they are not integrated.
    This certificate thesis evaluates the value proposition of an Internal Developer Portal, taking into account the
    challenges present in SBB IT.
    The findings were corroborated with a survey of DevOps engineers.  \\ \\
    The results of the analysis and the survey can be summarized as follows.
    \begin{itemize}
        \item DevOps engineers are largely satisfied with the current Internal Developer Platform
        \begin{itemize}
            \item The platform's continuous integration and continuous deployment capabilities were rated very positively
            \item The support for the lifecycle of resources was rated as very good
        \end{itemize}

        \item However, there are several findings with significance that could be addressed with an Internal Developer Portal
        \begin{itemize}
            \item The issue receiving the lowest rating pertained to financial transparency regarding the usage of platforms and resources (FinOps)
            \item DevOps engineers miss an integrated view of their application in the context of platform, resources and other applications
        \end{itemize}
    \end{itemize}

    In conclusion, this research suggests that, given its current technology and organizational landscape, SBB IT can
    generate value by investing resources in an Internal Developer Portal.
    This would serve to enhance the DevOps experience and reduce the cognitive load for engineers.
    The identified benefits would reduce operational overhead, improve software development in the areas of quality and
    deliverability, and provide faster resolution in the event of an incident.

    \pagebreak


    \tableofcontents
    \pagebreak

    \section*{Used Tools}
    In addition to the sources mentioned in the bibliography, chapter \ref{sec:bibliograhpy}, the following aids and tools have been used
    \begin{itemize}
        \item Version control with Git and GitHub, the source code for this document is at \url{https://github.com/thpeng/mas/tree/master/src/mca-dpl23-1}
        \item Deepl to support translations at \url{https://www.deepl.com/}
        \item ChatGPT 4 for proof-reading at \url{https://chat.openai.com/}
        \item Microsoft Forms to conduct a survey (\url{https://forms.office.com})
        \item The document is written in \LaTeX  (\url{https://miktex.org/}) with the editor IntelliJ IDEA (\url{https://www.jetbrains.com/de-de/idea/})
    \end{itemize}

    \section*{Declaration of Honour}

    I hereby confirm that I have
    \begin{itemize}
        \item prepared the present thesis independently and without the use of sources or aids other than those indicated,
        \item identified the sources used as such, either verbatim or in terms of content,
        \item not yet submitted this work in the same or similar form to an examination board.
    \end{itemize}
    Bern, \today\newline
    \newline
    \newline
    \newline
    \begin{tabular}{@{}p{5.0cm}@{}}
        \hrulefill \\
        Thierry Peng
    \end{tabular}

    \pagebreak

    % max 15 - 20 pages beginning here


    \section{Introduction}
    \label{sec:introduction}
    SBB has an extensive and steadily growing IT landscape of applications and tools to support the digitization of its
    core business processes.
    These digitization efforts brought major changes to the SBB IT department.
    In less than ten years, the development model has changed from a pure waterfall to an agile methodology,
    The identified benefits would reduce operational overhead, improve software development in the areas of quality and
    Cloud technologies have been adopted, replacing hardware-bound platforms.
    Additionally, a new internal organization model, based on the separation of business and line management, has been introduced.\\
    Even though these changes are now taking root and leading to various positive outcomes, they still pose a challenge.
    There is a new complexity with the rapidly changing technological landscape but also due to very many old,
    business-critical legacy systems.
    In addition, regulatory pressure has increased significantly in recent years, as exemplified by the
    HCBöV\footnote{Handbuch Cybersecurity für Betriebe des öffentlichen Vekehrs}, as well as the new Data Protection Act in Switzerland - nDSG.\\
    Feedback from various employees in different roles within the so-called ``Digitalen Zone`` at SBB indicates that the overall
    increase in complexity has an impact on the daily work of DevOps\footnote{DevOps is a combination of the words
    Development and Operations and is a set of practices} engineers.
    The picture emerges that although all information for the design, development and operation of
    applications is available in principle, it is maintained in all different places and forms.
    This makes the procurement of information for these activities time-consuming, requiring a DevOps engineer to be very
    familiar with the SBB IT landscape.

    \subsection{Problem Statement and Controversy}
    \label{subsec:iproblemstatement}
    Internet research and vendor guidance suggest growing momentum towards addressing these issues with an Internal
    Developer Portal (IDP) .
    The central promise of the concept of an Internal Developer Portal is that it can reduce the so-called ``cognitive load``
    of DevOps engineers.\\
    The Cognitive Load Theory is a framework used to describe individuals' learning ability and cognitive load when using
    different problem-solving strategies.
    Novices in a problem domain frequently apply general solution strategies, but these are so burdensome that they
    cannot develop and learn an efficient solution strategy at all\parencite{cogload}.
    Cognitive load in software development is experienced in different ways; intrinsic, extrinsic and germane load\parencite{cogdev}.
    An Internal Developer Portal claims to address the extrinsic load part with a one stop shop for information about teams,
    associated applications and their used infrastructure and resources for DevOps engineers.

    \subsection{Objective of the Certificate Work}
    \label{subsec:iobjective}
    The aim of this certificate thesis is to investigate the extent to which the aforementioned problems can be addressed
    through the introduction of an Internal Developer Portal.\\
    In a first step, the benefits of the concept of an Internal Developer Portal are elaborated and placed
    into the general context of software development and the operation of IT systems in the SBB.\\
    In a second step, the problem of complexity and lack of integration is underpinned with a quantitative survey.\\
    The third part consists in bringing together the results of the survey and the value proposition as well as making a
    recommendation for action.


    \section{Value Proposition for an Internal Developer Portal}
    \label{sec:vp}
    An Internal Developer Portal is a web-based information catalogue tailored to the needs of engineers involved in
    operating IT systems, developing software, and performing other related tasks.\\
    Internal Developer Portals are proposed as the solution for perceived problems\parencite{backstagestory} of DevOps
    engineers while developing and operating IT Systems such as
    \begin{itemize}
        \item A lack of a central repository for reliable and tailored information
        \item A high cognitive load of engineers due to switching between multiple and very different tools while developing or operating IT Systems
        \item Enhancing developer experience by abstracting the complexities of infrastructure management
    \end{itemize}
    One potential source of information which is displayed in an Internal Developer Portal is what is known as the Internal Developer Platform.

    \subsection{Internal Developer Platform}
    \label{subsec:vpplatform}
    Gartner distinguishes between the concepts of an Internal Developer Portal and an Internal Developer Platform:
    ``Internal Developer Portals serve as the interface through which developers can discover and
    access Internal Developer Platform capabilities``\parencite{gartner} .
    An Internal Developer Platform, according to the COO Christoph C. Richter from Humanitec\parencite{richteretal} ,
    should consist at least of the following component:
    \begin{itemize}
        \item Application Configuration Management
        \item Infrastructure Orchestration
        \item Environment Management
        \item Deployment Management
        \item Role-Based Access Control
    \end{itemize}
    Additional components, such as Observability, are also mentioned in other sources\parencite{xenon}.
    XENONSTACK as well as Humanitec mentioned before, are suppliers of Internal Developer Platform solutions with
    different capabilities.\\
    The capabilities and components of Internal Developer Platforms, as previously listed, are not exclusive to the
    products of these two suppliers.
    These components and practices are widely known and used in most contemporary IT operations or software development
    departments.\\
    Taking for example the practices of Application Configuration, Environment and Deployment Management:
    Beyond the size of a very small team, it is necessary to use tooling to know where, in which version
    and with what configuration a software artifact is installed.
    The reason for this may be to coordinate a new software rollout, to test the software on a dedicated stage, to fix bugs
    or to develop new features and to solve incidents caused by your software.\\
    Therefore, one could argue that the concept of an Internal Developer Platform is not unique to certain products or
    offerings, but rather applies to a set of tools and practices that support IT operations and software development.
    These tools and practices can be considered precursors to Internal Developer Platforms and are usually
    distinct solutions or products.\\
    For example there is a separate Configuration Management Database to address the need of Application Configuration
    Management or there is a purpose built CI/CD\footnote{CI/CD stands for continuous
    integration and continuous deployment and is a practice in software engineering. Its main focus is to automate all steps
    between development and rolling out on production, such as quality assurance, building artifacts and installing them.}
    pipeline for addressing the need for a Deployment Management.
    These solutions are not necessarily integrated with each other and may be products from different vendors.
    It may be also the case that these products or solutions were established before the advent of Agile Methodologies and DevOps.

    \subsection{DevOps}
    \label{subsec:devops}
    With DevOps the traditional disciplines of IT operations and software development are much more integrated to
    reduce friction, release features more frequently, and improve product quality\parencite{safedevops}.
    Before the advent of DevOps, different tools and processes were typically used for development and operations roles,
    largely because these roles were divided among separate departments or teams.
    There is a fundamental conflict in rigidly separating the roles of developer and operator.
    Developers were evaluated based on the speed of feature development and their responsiveness to changing requirements.
    Operators, on the other hand, were measured on how many failures their systems have and how quickly those can be patched.\\
    One of the basic principles of DevOps was coined by Werner Vogels, the current CTO of Amazon;
    ``You build it, you run it``\parencite{vogels}.
    This means that the same team that develops the software is also responsible for its operation.
    With DevOps, a team takes the full responsibility over the lifecycle of an application or platform.
    In combination with agile methodologies such as Scrum\footnote{Scrum is an iterative software development methodology}
    or SAFe\footnote{SAFe stands for Scaled Agile Framework and adds orchestration layers above agile (Scrum) teams for
    a better cross-team and product-centric alignment}, business-oriented individuals are also integrated into this
    cross-functional team.
    An example of a business-oriented role in a DevOps team is the product owner\parencite{safepo}.\\
    This also means that a team must have a wide range of skills and access to various tools for their work.
    Such tools can be used for activities such as requirements engineering, solution architecture, software development
    but also anything else needed to operate the software.
    Especially the operation of the software is a tool-intensive work.
    There are tools for logging, monitoring, incident handling, alerting and other practices.
    In the event of an incident, configuration management databases or enterprise architecture databases are crucial for
    identifying the team responsible for a particular application or resource.
    This information is important to determine the root cause of the incident and to quickly resolve it.
    Although DevOps offers numerous benefits, it is quite a complexity for a team member to master all necessary tools and
    practices in an agile DevOps team.
    Here, the Integrated Developer Portal offers several suggestions to help individuals and teams with their day-to-day operations.

    \subsection{Internal Developer Portal}
    \label{subsec:vpportal}
    Manjunath Bhat from Gartner argues that an Internal Developer Portal has three main characteristics\parencite{gartner}:
    \begin{itemize}
        \item Abstraction
        \item Developer-centric view
        \item Pluggable framework
    \end{itemize}
    The overarching goal of these features is to improve the DevOps experience by reducing the time it takes to search
    for information and to interpret it correctly.
    The search task is supported by a centralized software catalog.
    Interpretation of the data is supported by an underlying logical model that describes the assets, technologies
    and the relationships between them.
    The logical model varies among implementations of the Internal Developer Portal, but there are some common elements.

    \subsubsection{Vendors and Products}
    \label{sssec:vendors}
    While it is not the focus of this certificate work to delve into the details of each possible product on the market
    and make a comparison between them, it is worthy to describe the most relevant ones.\\
    Redpoints mentions five products as ``Universal Service Catalogue`` offerings\parencite{devportalsprimer}.
    This kind of Internal Developer Portal is not tied into specific cloud offerings such as Amazon Web Services, Microsoft Azure or Google
    Cloud Platform (``API Catalog tied to an API Gateway / Service Mesh``) and has no opinion about a particular application
    architectural style, such as the products mentioned in the category ``Microservice Catalog``.
    For further discussion about the capabilities of Internal Developer Portal the focus will be laid on this
    ``Universal Service Catalogue`` category, because it has the fewest conceptional restrictions.\\
    \begin{table}[!htbp]
        \begin{center}
            \begin{tabularx}{\textwidth}{XXXlX}
                \toprule
                Vendor   & Product      & Year & License               & Deployment  \\
                \midrule
                Spotify  & backstage.io & 2020 & open source           & self-hosted \\
                Lyft     & clutch.sh    & 2020 & open source           & self-hosted \\
                Moment   & moment.dev   & Beta & proprietary           & SaaS        \\
                Opslevel & opslevel.com & 2018 & partially open source & SaaS        \\
                \bottomrule
            \end{tabularx}
            \caption{\label{tab:vendors} Vendor and Products.}
        \end{center}
    \end{table}
    \FloatBarrier
    Roadie is omitted from the table \ref{tab:vendors} because it is a commercial offering of the software Backstage.
    It seems, there are two main directions.
    Backstage and Clutch, developed by two tech companies for their internal audience, have been made
    available as opensource.\\
    Moment and OpsLevel are purposefully built by startups as their product and have commercial offerings.
    For the further discussion about the value proposition of an Internal Developer Portal, some screenshots and concepts will
    be used from Backstage.\\
    Backstage, developed by Spotify, is a product of the company's heavy investment in what they refer to as the
    ``developer experience`` (DX)\parencite{spotifydx}.
    Additionally, in 2020, Backstage was accepted as an incubating project by the Cloud Native Computing Foundation\parencite{cncf} .

    \subsubsection{Discoverability and Obtaining Information}
    \label{sssec:disc}
    The Software Catalog is the main feature of an Internal Developer Portal.
    Its purpose is to make the capabilities of the Internal Developer Platform discoverable for an engineer as seen in the
    screenshot in figure \ref{fig:catalog}.\\

    \begin{figure}[h]
        \includegraphics[width=\linewidth]{backstage_catalog}
        \caption{The Software Catalog from the Backstage Demo\parencite{backstagedemo}}
        \label{fig:catalog}
    \end{figure}
    The content of the catalog may be any combination of
    \begin{itemize}
        \item Teams with members
        \item Applications
        \item Infrastructure and platforms
        \item Most importantly, the relationships between those.
    \end{itemize}
    Data may be shown in a tabular fashion or as a relationship graph.
    The catalog allows a search over multiple entities.
    Usually, the user's context is automatically loaded.
    For example, the application they are responsible for appears on the initial screen.
    Users can bookmark their favorite items, and their own resources are marked as ``owned``.
    In addition to the tabular overview provided by the catalog, for each component exists a detail view, as shown in
    figure \ref{fig:portaldetails}.
    \begin{figure}[h]
        \includegraphics[width=\linewidth]{backstage_item_details}
        \caption{Details about an artifact}
        \label{fig:portaldetails}
    \end{figure}
    \FloatBarrier
    In the details view more attributes are shown for the chosen item from the overview.
    Depending on the kind of item, different attributes are shown.
    For an application, it may be its used resources, consumed and provided APIs and other subcomponents.
    For teams, the members are shown with their owned resources and also the details how to contact the team is available.

    \subsubsection{Platform as a Product}
    \label{sssec:paap}
    In some implementations of an Internal Developer Portal, the catalog is not just a copy of an architecture database.
    In addition to the aforementioned applications, platforms, organizational data and resources, other products
    of the teams are represented.
    For example, platform teams may provide support and consulting services for their platform or area of expertise and offer
    this service.
    Some teams use the Internal Developer Platform to showcase code or libraries solicit input or to create an
    inner-source community.\\
    In an expertise-sharing session between SBB and Booking.com, Booking.com reported\parencite{bookingcom} its Internal
    Developer Portal as a standalone product that is managed by a dedicated team.
    This team is also responsible for on-boarding other platform and application teams, convincing them of the
    value of an integrated Internal Developer Portal for the benefit of Booking.com.

    \subsubsection{Golden Path}
    \label{sssec:goldenpath}
    The term ``Golden Path`` was coined by Spotify\parencite{spotifygoldenpath} and is itself a reference to Frank
    Herbert`s Dune trilogy.
    The meaning of this term is that there is only one viable path to do things the right way and to succeed.
    A Golden Path in the context of an Internal Developer Portal is a ready-to-use template that enables quick project startup.
    It can consist of software engineering artifacts and preconfigured infrastructure components and is usually very
    opinionated.
    Opinionated means that the context of the enterprise is already considered, such as internal regulations,
    architectural guidelines, the currently supported tech stack, and other internal specifications.
    For example, a golden path for a database is already built into the company's security infrastructure,
    logging is ensured, and some best practices for database configuration and modeling are already implemented.\\
    Golden Paths are not necessarily product-specific;
    they can represent a typical development stack used within the enterprise.
    For example, a stack for a typical web application may consists of a frontend built with Javascript, a Java backend
    and an Oracle database, all integrated into a single build pipeline for CI/CD,    securely configured and with
    the enterprise look and feel already established.
    With such a template mechanism, new applications can be built faster and the same solutions are used for multiple
    applications.
    Additionally, the developers do not have to write tedious code to integrate each of these components with each other
    and can focus quickly on the business logic they have to implement.

    \subsubsection{Technical Documentation and Communication}
    \label{sssec:techdoc}
    Platforms that use an Internal Developer Portal may choose to migrate their user-facing documentation to the portal.
    The basis for this technical documentation can be a solution based on Markdown\parencite{backstagetechdocs}
    or a similar syntax.
    The advantage of this approach is that the documentation is treated like code and can be integrated into the same toolchain.
    For example, the documentation sits alongside the platform automation code, is peer reviewed, created
    and published in sync with the platform's update cycle.\\
    In the discussion mentioned in the chapter \ref{sssec:paap}, Booking.com discussed an integration feature in their
    Internal Developer Portal that incorporates communication about the lifecycle and changes to their platforms.
    With this approach it is possible to inform DevOps engineers with targeted communication, especially combined with data about relationships
    between platforms, team resources and application as described in chapter \ref{sssec:disc}.
    This is especially important when communicating breaking changes\footnote{A breaking change is a change in a software
    which makes an action necessary by the user or a client. For example, a new release of a software needs a newer version
    of an operating system to function properly.}, release notes or information about outages.

    \subsubsection{Making Expertise Visible}
    \label{sssec:expertise}
    Splunk\footnote{Splunk is a company specialised in creating software for log analysis and monitoring and it is known
    for its product with the same name}
    describes a use case demonstrating how an Internal Developer Portal can be leveraged to make expertise more visible
    within their company\parencite{splunkidp} .
    One challenge is that it is usually not transparent within the company which engineer possesses what kind of specialty
    know-how.
    For example, one team integrated successfully a NoSql\footnote{NoSql databases are a broad class of databases which
    are not using the classic Structured Query Language (SQL). Examples are key/value stores, document oriented
    databases and others.  } database.
    This database is not part of a platform service but an integral part of a business application.
    The next team that wishes to integrate the same technology faces the issue of lacking an internal expert in this
    technology and may not be aware that another team has already mastered it.
    Splunk's solution is to enrich teams and users in the Internal Developer Portal with additional data about their
    knowledge and allow other users to provide feedback on how a team or user supported them.
    They refer to this feedback as a ``Bravo`` .
    Thus, they create something like a social network inside their company for connecting people to solve problems or
    easing the way in how to find people with specific skills or knowledge.

    \subsubsection{Extendability of the Platform}
    \label{sssec:extendability}
    As illustrated in chapter \ref{sssec:vendors}, there are multiple offerings of Internal Developer Portals and a
    variety of implementations.
    In case of the Software-as-a-Service offerings, the extendability consists of using plugins for well-known
    technologies or products and preparing and inserting the company's data for the catalogue feature.
    Self-hosted variants, like Backstage and Clutch, function more like a skeleton or a framework.
    For both products, there are pre-built binaries which are installable and runnable, but you can choose to build your
    own Internal Developer Portal and use these products as a starting point.
    Both support custom-made plugins on the front-end, which can utilize additional data sources outside the immediate
    realm of the Internal Developer Portal.
    This approach is handy for decentralized organizational structures where each team has maximum autonomy over how
    they achieve their goals, but also wishes to utilize and contribute to a centralized catalog by adding data and
    functionality via plugins.
    The plugin architecture allows to leverage your Internal Developer Portal with the tooling landscape in a
    single and integrated user experience for DevOps Engineers.

    \subsubsection{FinOps}
    \label{sssec:finops}
    FinOps is a practice to enable DevOps Teams to make decisions about resource usage and creates transparency
    about the costs incurred by these decisions.
    The FinOps foundation defines\parencite{finopsdefinition} this practice as ``FinOps is an evolving cloud financial
    management discipline and cultural practice that enables organizations to get maximum business value by helping
    engineering, finance, technology and business teams to collaborate on data-driven spending decisions.``.
    An Internal Developer Portal may help with this practice by providing functionality for one of the basic disciplines
    of FinOps: ``Understanding Cloud Usage And Cost``.\\
    The Backstage.io Demo\parencite{backstagedemocost} shows one possible implementation.
    This capability enables a DevOps team to take responsibility also in the financial domain and creates transparency
    for the business what the true costs of their product is.


    \section{Situation at the SBB IT}
    \label{sec:sbbit}
    The IT department of the Swiss Federal Railways (SBB) consists of around 1300 specialists and is responsible for
    700 applications\parencite{sbbitkennzahlen}.
    The applications built and operated by this department are used for a wide range of use cases.
    For example, some applications are covering generic enterprise use cases such as human resource management, finance
    and controlling.
    Other software is specially built for mission-critical, day-to-day operations of the railway such as the Rail Control
    System (RCS)\parencite{sbbrcs} .\\
    An SBB IT application might be a purchased product, developed in-house, or a combination of both, sometimes referred
    to as a ``customization``.\\
    Applications and products, built and operated by SBB, adhere to various standards and regulations set by the Swiss
    government and industry initiatives.
    Examples are the EN 50128 standard by the European Committee for Electrotechnical Standardization (CENELEC)\parencite{cenelec},
    which governs among other things the development of safety-related software with different requirement levels.
    Another example is HCBöV\parencite{hcboev}, a swiss government regulation concerning cyber-security, risk management,
    business continuity management among others.\\
    In addition to the more business-oriented applications, there is a need for basic platforms, tools and services to help create
    and operate these applications.
    Examples for a foundational platform is a PaaS\footnote{A platform-as-a-service (PaaS) is a software which exposes
    resources such as compute or storage to applications. The team which is building an application is relieved from the
    burden of obtaining and provisioning resources on a hardware level. A well-known implementation of a PaaS is Kubernetes from the CNCF.}.
    This platform and other services are provided in the SBB by the so called DSRVs\footnote{
        A DSRV stands for Digital Service and is one or more teams with multiple services and platforms. A DSRV is built
        around a domain, eg. Monitoring, Cloud, Application Integration and others}.

    \subsection{Internal Developer Platform}
    \label{subsec:sbbplatform}
    SBB IT already has solutions in place for most of the core components of an Internal Developer Platform, as laid out
    in Chapter\ref{subsec:vpplatform}:
    \begin{itemize}
        \item Application Configuration Management - is centered around the practice of GitOps\parencite{hashicorpvault}
        \item Infrastructure Orchestration - for example, SBB uses OpenShift since 2015\parencite{rhsbbopenshift}
        \item Environment Management - provided by OpenShift and CI/CD Pipelines
        \item Deployment Management - is provided by CI/CD pipelines built on Tekton and ArgoCD\parencite{sbbtekton}
        \item Role-Based Access Control - built-in in most of the SBB IT platforms
    \end{itemize}
    In addition, there are several other tools and applications used for the day-to-day operations and development.
    Most notable are the following:
    \begin{itemize}
        \item Development and operation are dependent on a Wiki Software for documentation and communication.
        \item An Enterprise Architecture Database contains the information about used technologies, platforms and
        dependencies, modelled by IT architects
        \item An IT Service Management Platform contains the data concerning team contact information and support groups for applications
        \item A Configuration Management Database contains data about used assets, resources and relations to applications
    \end{itemize}
    Thus, while it could be argued that the foundation for an Internal Developer Platform is in place, there are still
    two critical pieces missing.
    First, most of the components mentioned are only partially integrated with each other.
    Examples for missing integration are different identifier for assets, unaligned authorization such as per-user vs.
    per-group models and manual data maintenance.
    Second, there is no central view to aggregate data contained in these applications and tools for the benefit of
    the DevOps engineers.

    \subsection{DevOps}
    \label{subsec:sbbdevops}
    In 2015, DevOps practices and a PaaS were introduced in SBB IT .
    The stated goal of these additions was to improve the increasing complexity in operations and the perceived slow
    speed in responding to change\parencite{sbbdevops}.\\
    Four years later, a reorganization introduced a new operating model to SBB IT, based on agile principles, DevOps,
    and SAFe, resulting in the creation of the BizDevOps engineer role\parencite{sbbagile} .
    Most of the existing engineering roles, such as the Application Engineer, Operations Manager, Test Engineer or
    Requirements Engineer were transformed into this BizDevOps Engineer role.
    To bridge the gap between business and IT and break down silos, some business-facing roles have also been integrated into these
    cross-functional teams have been integrated and now have the title of a BizDevOps engineer.\\
    By 2023, DevOps teams have become the established standard in SBB IT.
    Despite this, they still work primarily with tools for development and operations from the predecessor organization.
    The tools themselves may be state-of-the-art in their areas, but they were not purchased or developed with
    end-to-end integration in mind.

    \subsection{Internal Developer Portal}
    \label{subsec:sbbportal}
    Most BizDevOps teams at SBB make do by structuring of their important information for daily work, to cope
    with the lack of a single and integrated portal.
    A common approach is to use the internal wiki software where each team maintains their relevant http link collection
    related to their project on one or more pages.
    In most cases, when someone joins a team, their first task is to familiarize themselves with this collection.
    Of course, these link collections become outdated and maintaining them also requires effort from a team.
    When entire teams are newly staffed, for example with models such as managed capacity or near-shore, it is even more
    difficult for them to get all the information they need.\\
    While this discovery and knowledge sharing issue has not been fully resolved, there have been some efforts to create
    a self-service portal for ordering resources.
    The reason for this portal was that the platform teams were dissatisfied with the unstructured ordering of resources
    via email, phone, or other ad hoc methods.
    So a modular portal was built that allowed each team to integrate their ordering process.
    However, there was no single owner of the portal, and after the immediate pain point of each platform team was
    resolved, there was little effort to expand the functionality to include even more use cases.


    \section{Method}
    \label{sec:method}
    The method chosen to verify the observations about the state of the DevOps Experience and to evaluate the value proposition
    of an Internal Developer Portal, is to hold a survey.
    The survey is created with a customer journey based on different BizDevOps personas.
    The personas and the customer journey are based on a preliminary, internal work\parencite{sbbjobstobedone}.

    \subsection{Customer Journey}
    \label{subsec:cusjour}
    The customer journey is focused on the interaction of different personas with the existing Internal Developer Platform.
    The assumption is that this interaction between the BizDevOps Engineers and the Internal Developer Platform has a big
    impact on the DevOps Experience.\\
    The personas of the customer journey are roles which are quite common in a BizDevOps team in the SBB IT.\\
    \textit{Peter Plattform} is working as a platform engineer and is interested what the platforms are offering for building
    an application.
    Peter orders a resource from a platform and is responsible for the full lifecycle until it this resource is not
    needed anymore.\\
    \textit{Diego Developer} is a software engineer and his duty is to work on features in the context of the application
    he is responsible for.
    He needs information about the current tech stack\footnote{A tech stack is short for technology stack.
    A technology stack is usually a selection of available components and framework such as databases, messaging, programming
    languages and other technologies necessary to build applications.} and he needs advice for the solution design of his application.
    Diego is also interested to get the releases of his application as fast as possible and with the highest quality
    into the production environment.\\
    \textit{Norbert Neuling} is the rookie in the team and started two days ago in this team.
    He needs first to install the necessary tooling to enable him to develop features.
    The next assignment for him for his learning journey is to create a sample application on OpenShift and integrate
    it with sample database provided.
    Additionally, Norbert needs some guidance about what the best practices are concerning the used technologies and platforms.\\
    \textit{Agnes Architekt} is working as an IT architect in the team.
    The product owner of the application, for which she is responsible, asked her to identify possibilities to lower operational costs.
    She is also responsible for the enabler\footnote{An enabler is a work item which has no direct business value.
    For example, a migration to a new version for a web framework or other technical necessities.} of the application.
    She has to create the necessary backlog items and communicates with the product owner, so that he can prioritize
    them and put them in one of the next product increments\footnote{SAFe terminology for a cycle of usually five two-week SCRUM Sprints of agile software development.}.
    For this work, she needs to know about the upgrade plans of the used services of the platforms her application is using.     \\
    \textit{Ida Incident} is working on an operational issue which is affecting the users of her application.
    She needs to know the technical topology of her application and the platforms which are used for operating this application.
    She needs to identify the team responsible for the given platform and understand how she can get in touch with them.\\
    For more information about the customer journey, see figure \ref{fig:customerjourney}.

    \subsection{Quantitative Survey}
    \label{subsec:quansur}
    For a quantitative survey, a mix of questions were created based on the customer journey.
    Some relate to the value proposition of an Internal Developer Portal and others to other known or suspected issues
    related to the existing platform for internal developers.
    The survey is intentionally short, as longer surveys typically tend to show low response rates among BizDevOps engineers.
    The target audience for the survey is SBB's BizDevOps engineers around SBB it is cloud platforms.
    One of the main sources of information for BizDevOps engineers is the CLEW\footnote{CLEW stands for Cloud, ESTA,
        itself an acronym of development stack and WZU, tool support} blog.
    This blog has a total of 780 subscribers.
    The survey itself was created using Microsoft Forms.
    The start date of the survey was 24.04.2023 and the end date was 08.08.2024.

    \subsubsection{Questions}
    \label{sssec:questions}
    Seventeen rating questions were created from the steps of the Customer Journey.
    The rating questions are documented in the appendix in the table \ref{tab:ratingquestionstable}.
    The correlation between the proposed values in chapter \ref{sec:vp} and the questions is as follows:
    \begin{itemize}
        \item Questions 1, 13, and 14 relate to Discoverability and Obtaining Information in chapter \ref{sssec:disc}
        \item Question 2 refers to Making Expertise Visible from chapter \ref{sssec:expertise}
        \item Questions 3 and 17 relate to Technical Documentation and Communication, chapter \ref{sssec:techdoc}
        \item Question 10 refers to chapter \ref{sssec:goldenpath}, Golden Path.
        \item Question 12 refers to FinOps, chapter \ref{sssec:finops}
    \end{itemize}
    Questions 4, 11, and 16 do not closely correlate to the concept of an Internal Developer Portal.
    However, these three questions are related to general capabilities of the existing Internal Developer Platform.\\
    Questions 5, 6, 7, 8, and 9 measure the maturity of Lifecycle of Resources capabilities.\\
    There are no direct questions about the Extendability \ref{sssec:extendability} chapter and Platform as a Product
    \ref{sssec:paap}.
    Extensibility is a non-functional requirement when choosing a specific solution for the Internal Developer Portal.
    This feature is not directly relevant or desired by the portal's users.\\
    Platform as a Product is a goal of SBB IT internal product management to make the work of BizDevOps
    team more visible and accessible and is not an intrinsic goal of an engineer.
    No question in the survey addresses this topic.\\
    Question 15 asks about support in the incident process and does not refer to an Internal Developer Platform or an
    Internal Developer Portal. \\
    The survey and questions relate to DevOps experience and use of the platform, products, or services of the so-called CORE,
    which consists of multiple DSRVs.
    Most of these DSRVs operate mission-critical platforms.
    It is not required that the respondent has used all services and platforms of the CORE.
    It is also scoped about usage of services and platforms in the last three months.\\
    The rating questions have choices from one to five stars and are mandatory to complete.
    The meaning of the rating as defined as follows:
    \begin{itemize}
        \item One star - non-existent or unknown
        \item Two star - with room for improvement
        \item Three star - I can work with it, corresponds to what I expect
        \item Four star - above average
        \item Five star - super that is what makes you
    \end{itemize}

    In addition, two other questions were asked which are documented in table \ref{tab:oetable}.
    These questions are primarily for the internal needs of the product management of SBB IT platforms.
    They provide feedback on overall satisfaction and allow respondents to make additional points.
    One is a question with a that allows free text responses.
    The goal is to get additional input and to generate interview partners for a qualitative survey.
    This question is optional for the respondents.
    The outcomes from these interviews are beyond the scope of this certificate thesis.\\
    The final question\footnote{The engagement question was added based on input from the internal product management team}
    is an engagement question and is intended to measure the overall satisfaction of BizDevOps engineers related to the CORE platforms and services.
    % 2.3.2 Discoverability -> 1, 13, 14 (strong)
    %2.3.3 Platform as a Product n/a
    %2.3.4 Golden Path -> 10 (strong)
    %2.3.5 Technical Documentation, -> 3,17 -> ev. noch Extend! (strong)
    %2.3.6 Expertise -> 2, (strong)
    %2.3.7 Extendability
    %2.3.8 FinOps -> Frage 12. (strong)
    % n/a -> 4, 11, 16 (platform)
    % Lifecycle: -> 5, 6, 7, 8, 9 (weak)
    % monitoring, -> 15 (none)

    \subsubsection{Result of the Rating Questions}
    \label{sssec:rratque}
    After the two-week survey period, the response rate was 9.74\% with 76 respondents and a margin of error of 11\%.
    The raw results for the rating questions are included in table \ref{tab:rawratingquestionresultstable}.
    The calculated scores for the rating questions are included in table \ref{tab:ratingquestionresultstable}.
    On average, the seventeen questions have a mean score of 3.40, with 18.5\% of respondents giving negative responses
    (1 and 2), and 51.3\% giving positive responses (4 and 5)
    Overall, based on an industry standard 95\% confidence level\parencite{nistmean}, a population size of 76, a critical value
    of 1.96\parencite{ci} and a standard deviation ($ \sigma $ ) of 1.0155, the confidence interval (CI) is 3.40 \textpm 0.228.\\
    % low: 3.172, high: 3.628
    % calc \[ 0.228 = 1.96 * (1.0155 / \sqrt{76})]
    %see https://www.scribbr.com/statistics/confidence-interval/
    Table \ref{tab:aggregateidpresults} documents the aggregate result for the five value propositions along with the eight associated questions.\\

    \begin{table}[!htbp]
        \begin{center}
            \begin{tabularx}{\textwidth}{lXXXXXX}
                \toprule
                Topic           & Ids       & Mean & Negative & Positive & $ \sigma $ & CI            \\
                \midrule
                Discoverability & 1, 13, 14 & 3.11 & 24.4\%   & 35.1\%   & 0.932      & \textpm 0.121 \\
                Expertise       & 2         & 3.64 & 12.0\%   & 65.3\%   & 1.055      & \textpm 0.237 \\
                Documentation   & 3, 17     & 3.39 & 18.7\%   & 51.9\%   & 0.956      & \textpm 0.152 \\
                Golden Path     & 10        & 3.21 & 20.0\%   & 42.7\%   & 0.929      & \textpm 0.209 \\
                FinOps          & 12        & 2.62 & 45.3\%   & 21.3\%   & 1.200      & \textpm 0.270 \\
                \bottomrule
            \end{tabularx}
        \end{center}
        \caption{\label{tab:aggregateidpresults} Aggregated results for IDP related questions.}
    \end{table}
    \FloatBarrier

    In contrast, the aggregated results of questions on other topics are contained in the table \ref{tab:nonidpresults}.
    The results are grouped by resource lifecycle questions, generic questions about the Internal Developer Platform
    and incident handling.\\
    \begin{table}[!htbp]
        \begin{center}
            \begin{tabularx}{\textwidth}{llXXXXX}
                \toprule
                Topic     & Ids           & Mean & Negative & Positive & $ \sigma $ & CI            \\
                \midrule
                Lifecycle & 5, 6, 7, 8, 9 & 3.59 & 13.9\%   & 59.5\%   & 1.090      & \textpm 0.110 \\
                Platform  & 4, 11, 16     & 3.46 & 15.6\%   & 54.7\%   & 1.038      & \textpm 0.134 \\
                Incident  & 15            & 3.92 & 10.7\%   & 74.7\%   & 1.055      & \textpm 0.237 \\
                \bottomrule
            \end{tabularx}
        \end{center}
        \caption{\label{tab:nonidpresults} Aggregated results for non-IDP related questions.}
    \end{table}
    \FloatBarrier

    \subsubsection{Result of the Open Question}
    \label{sssec:ropque}
    32 of the 76 respondents answered the open question.
    Due to the sensitive nature of the answers regarding the technologies, teams and processes used, only some excerpts
    of the answers are reproduced in this chapter.
    The answers in full are documented in the internal wiki\parencite{sbbdevopsexperience}.\\
    Six of the respondents mentioned the topic of Technical Documentation and Communication.
    One BizDevOps engineer would like to have ``Information und Dokumentation zentralisierter oder verknüpft.``.
    Another mentioned the decentralized communication about platform news: ``Man müsste eigentlich jemanden separat einstellen
    der all die Blogs etc. pp. im Auge behält. Es passiert immer wieder, dass irgendwo eine Information zwischen die
    Stühle fällt, weil es einfach aufgrund der Menge untergeht.``\\
    Four people mentioned the CI/CD pipelines.
    The older, but still supported, CI/CD pipeline was mentioned several times regarding issues. \\
    Three respondents had some suggestions for additional tools in the development tools installer and additional
    feedback for it.
    This topic was covered in question 4.
    One respondent was very pleased and said about the setup process for development tools: ``In 15 Minuten
    Entwicklungsumgebung aufsetzen und los coden ist schon geil! :-)``\\
    Three respondents simply stated their gratitude and thanked the various service and platform teams of the CORE for
    their effort.
    For example, one respondent simply wrote ``danke, von Herz``.\\
    Two respondents mentioned the topic of financial transparency, covered by the question 12.
    One quote about this topic by an BizDevOps engineer was: ``Kosten sind sehr schwierig zu verstehen und nachvollziehen auf die Rechnung. Reporting
    sollte unbedingt verbessert sein, man weiss nicht wo man könnte optimieren (zu beispiel welche anteil von kosten für
    xyz kommt von storage? von anzahle objekte? usw.)``\\
    There were also two suggestions about missing infrastructure as code capabilities.\\
    Two feedback concerned the current vision and its wording of the CORE DSRVs.\\
    Two questioned a perceived lack of consultancy and unified support.\\
    And finally, there were two critical votes about the modernization effort of the SBB IT which is known as the
    ``Gandalf Plan`` and the perceived primacy of technology enabler over features for the business.\\
    Six feedbacks each concerned different other topics which will not be discussed in more detail.
    % techdoc + Benach: IIIIII
    % CI/CD kritik: IIII
    % Discovery: IIII
    % Enhancement local installation: III
    % thanks: III
    % finOps: II
    % infrastructure-as-code missing: II
    % critic of vision: II
    % missing consultancy / unified support: II
    % gandalf kritik: II
    % multiple tools for the same job: I
    % Availability of toolchainL: I
    % non-informatiker: I
    % missing review/freigabeprozess: I
    % missing resource lifecycle for delete, upgrade: I
    % missing devsecops: I

    \subsubsection{Result of the Engagement Question}
    \label{sssec:rengque}
    The average result of the engagement question was 7.70 with a standard deviation of 1.452.
    The distribution can be seen in figure \ref{fig:engque}.
    \begin{figure}
        \includegraphics[width=\linewidth]{engagement.PNG}
        \caption{Results of the Engagement Question}
        \label{fig:engque}
    \end{figure}
    The result of the engagement question is not relevant for evaluating the value proposition of an Internal Developer Portal.
    For the product management team of the SBB IT, it is valuable to gauge the overall net promoter score\parencite{nps}
    of the CORE platforms and services.
    \begin{itemize}
        \item Values between 1 to 6 are considered detractors; fourteen responses fell into this category
        \item Values between 7 and 8 are considered passives; forty-one responses fell into this range
        \item Values between 9 and 10 are considered as promoters; twenty-one responses fell into this category
    \end{itemize}
    With the NPS method applied, there are 18.42\% detractors and 27.63\% promoters which amounts to a net NPS score of
    9.21.


    \section{Conclusion}
    \label{sec:conclusion}
    The evaluation data from chapter \ref{sssec:rratque} indicates that the topic of FinOps is significantly below the
    survey's mean of 3.40.
    FinOps has the most respondents with a negative attitude (45.3\%).\\
    Considering the confidence interval, Discoverability and Obtaining Information (3.11) is below the mean of the survey.
    Additionally, this topic has the second highest amount of respondents with a negative attitude in question 14 of 37\%.\\
    The mean of the Documentation and Communication theme is 3.39, almost on par with the overall mean and well within the
    confidence interval.
    The topic Making Expertise Visible has an average score of 3.64 which is above the overall mean.
    Topics not directly related to the Internal Developer Portal are at or above the overall mean.
    The general Internal Developer Platform topic has an average score of 3.46 which is near the overall average.
    With an average score of 3.59, the topic of Lifecycle of Resources, considering the confidence interval, is above
    the overall mean of the survey.
    The topic Incident Support topic has an average of 3.92 and is significant above the overall mean.\\
    The data can be summarized as follows:
    \begin{itemize}
        \item Two of the five tested value propositions of an Internal Developer Portal are below the overall average and have no overlap of the confidence interval
        \item One of the three other topics are above and two are near the overall mean of the survey
        \item Fifteen answers for the open questions contain issues related to the value propositions of an Internal Developer Portal
    \end{itemize}
    Thus, there is an argument that an Internal Developer Portal could be of value for the BizDevOps engineers if the following
    criteria are met:
    \begin{itemize}
        \item The Internal Developer Portal must address the topic of FinOps
        \item The Internal Developer Portal must improve the discoverability of services, teams, resources and platforms
    \end{itemize}
    For the topics Golden Path and Technical Documentation and Communication, the picture is less clear.
    Both have an overlap of their confidence interval with the overall mean of the survey.
    The topic Golden Path rated higher than the Documentation and Communication, but interestingly the Technical Documentation and
    Communication topic yielded six responses with the open question.
    It is possible, but not certain, that both of these topics may benefit from an Internal Developer Portal.
    There seems no current need to use an Internal Developer Portal for making Expertise more visible as this topic is
    currently of no concern.\\
    In contrast, it seems that the non-IDP topics do not have an immediate need for action.
    There are isolated requests for improvement, mostly centered around the old CI/CD pipeline and the local installation
    of development tools.\\
    Interestingly, the need to invest in ordering, modifying and deleting resources,
    which is often mentioned internally by the product management of the SBB IT, is refuted.

    \subsection{Recommendations for Action}
    \label{subsec:arec}
    The recommendations for action are for SBB IT with the current status in May 2023 regarding the Internal Developer Platform.

    \subsubsection{Focus on Discoverability}
    From the evaluation questions to the open-ended question responses to additional direct feedback, a universal criticism emerges.
    There is no capability to get an integrated view of applications, teams, platforms and resources in the specific context of a BizDevOps engineer.
    There is currently no platform of this kind and could be the unique selling point of an Internal Developer Portal in the SBB IT.

    \subsubsection{Improve the DevOps Experience}
    Targeting the needs of the BizDevOps Engineers is valuable.
    They currently lack a coherent view of operating applications, and improvements can potentially lead to immediate
    positive effects, such as a decrease in Key Performance Indicators (KPIs) like MTTR\footnote{MTTR stands for mean time to repair. Its the measured time between a failure of a system
    until it is fixed again and in working order.}, or an improvement in the
    net promoter score in a future survey.
    Initially, avoid targeting roles other than BizDevOps Engineers, such as management, architects, and security teams,
    with an Internal Developer Portal.
    This strategy will prevent dilution of the portal's focus and will maximize its impact on the BizDevOps engineers' work.
    If necessary, try to onboard these additional users and use cases later on.

    \subsubsection{Do not Try to Replace Existing Tooling}
    It is tempting to use an Internal Developer Portal as an opportunity to simplify and consolidate the existing tooling
    landscape, for example the components of the Internal Developer Platform.
    Remember, the Internal Developer Portal is only the frontend and makes the Internal Developer Platform accessible
    as seen in chapter \ref{subsec:vpplatform}.
    Most of the presented vendors and products in chapter \ref{sssec:vendors} have the capability and the frameworks
    necessary to achieve this separation of concern.

    \subsubsection{Prioritize Missing Capabilities}
    Try to prioritize missing capabilities like the mentioned FinOps instead of building again well understood and already
    implemented features such as ordering a resource.

    \subsubsection{Use Runtime Data as the Foundation}
    There is no need to build an enterprise application database again.
    IT Architects are using the established enterprise application database for modelling their solutions and thus it
    may not reflect the current state of an application and is less usefully in the incident case.
    Focus instead on the runtime data.
    For example, use data such as which application is deployed on a platform, telemetry data of applications or platforms or
    configured resources if GitOps\footnote{
        GitOps is a practice where resource definitions are stored in a source code version system. If any change
        happens in this definitions, the values are synced to a runtime, for example to a Kubernetes cluster.} is used.

    \subsubsection{Discover Excitement Features}
    The survey was constructed to identify primarily the needs of the BizDevOps engineers.
    Unsurprisingly, the survey did not identify any Excitement Features.
    It might be possible to identify potential Excitement Features with another methodology which would potentially lead
    to a better and faster adoption of an Internal Developer Portal.

    \pagebreak
    % max 15 - 20 pages end here


    \section{List of Sources}
    \label{sec:bibliograhpy}
    \printbibliography[heading=none]


    \section{Appendix}
    \label{sec:appendix}

    \subsection{Customer Journey Map}
    \label{subsec:cusjourmap}
    \begin{landscape}
        \begin{figure}[h]
            \includegraphics[origin=c,width=0.9\linewidth]{customer-journey.png}
            \caption{Customer Journey Map}
            \label{fig:customerjourney}
        \end{figure}
    \end{landscape}

    \subsection{Rating Questions}
    \label{subsec:rating}
    \begin{table}[!htbp]
        \begin{center}
            \begin{tabularx}{\textwidth}{lX}
                \toprule
                Id & Question                                                                                                                                                                              \\
                \midrule
                1  & Auffindbarkeit von Informationen über die Plattformen und Services. Zb. Confluence, Ardoq, Intranet...                                                                                \\
                2  & Ich kriege schnell und unkompliziert Beratung bei der Konzipierung von Applikationen, zb. für Engineering oder Development.                                                           \\
                3  & Qualität der Dokumentation der Plattformen und Services (Sowohl Umfang, Aktualität und Inhalt).                                                                                       \\
                4  & Onboarding, bis ich meine Werkzeuge für Dev oder Ops lokal installiert habe.                                                                                                          \\
                5  & Lifecycle von Infrastruktur im Selfservice, hier; ich kann Services und Plattformen schnell und unkompliziert ausprobieren.                                                           \\
                6  & Lifecycle von Infrastruktur im Self-Service, hier Bestellen von Plattformen                                                                                                           \\
                7  & Lifecycle von Infrastruktur im Self-Service, hier, Upgrade von Plattformen / deiner Applikation auf den Plattformen                                                                   \\
                8  & Lifecycle von Infrastruktur im Self-Service, hier Modifikation von Plattformen oder Applikationen.                                                                                    \\
                9  & Lifecycle von Infrastruktur im Self-Service, hier Run-Down oder Rückgabe von Plattformen                                                                                              \\
                10 & Es gibt einen Blueprint / Golden Path / Vorlage, die mir zeigt, wie ich eine Plattform verwenden kann.                                                                                \\
                11 & Die Plattformen erlauben mir, meine Ressourcen automatisiert zu bauen und zu deployen (CI/CD)                                                                                         \\
                12 & Ich kann meinem PO die Kosten unserer genutzten Services und Plattformen nachvollziehbar erklären.                                                                                    \\
                13 & Ich verstehe, wie meine Applikation mit den Plattformen zusammenhängt bei Incidents                                                                                                   \\
                14 & Welches DSRV Team für welche Plattform verantwortlich ist, ist mir klar.                                                                                                              \\
                15 & Ich kriege schnell und unkompliziert Hilfe im Incident Fall.                                                                                                                          \\
                16 & Die Plattformen zeigen mir an, ob sie funktionsfähig sind oder nicht. (zb. Monitoring, Chat, ...).                                                                                    \\
                17 & Ich, bzw. mein Team, werden rechtzeitig und mit genug Details informiert, wenn sich auf den Plattformen etwas ändert, was Auswirkungen auf die von uns verantwortete Applikation hat. \\
                \bottomrule
            \end{tabularx}
        \end{center}
        \caption{\label{tab:ratingquestionstable} Rating questions used in the survey.}
    \end{table}

    \subsection{Open and Engagement Questions}
    \label{subsec:openandengagement}
    \begin{table}[!htbp]
        \begin{center}
            \begin{tabularx}{\textwidth}{lX}
                \toprule
                Id & Question                                                                                                                                                                                                                                                                      \\
                \midrule
                18 & Was du uns sonst noch mit auf den Weg geben willst für die ``geilste DevOps Experience``, die du erleben möchtest. Falls du bereit bist, in einem kleinen Interview noch mehr Auskünfte zu erteilen und Nachfragen zu beantworten, gib uns doch hier auch deine Kontaktdaten. \\
                19 & Auf einer Skala von 1 \- 10, würdest du CORE (die genannten DSRVs und deren Plattformen) deinen Freunden und Arbeitskollegen weiterempfehlen?                                                                                                                                 \\
                \bottomrule
            \end{tabularx}
        \end{center}
        \caption{\label{tab:oetable} Open and engagement questions used in the survey.}
    \end{table}
    \FloatBarrier

    \subsection{Rating Questions Results}
    \label{subsec:ratingresults}
    \begin{table}[!htbp]
        \begin{center}
            \begin{tabularx}{\textwidth}{lXXXXX}
                \toprule
                Id & One Star & Two Star & Three Star & Four Star & Five Star \\
                \midrule
                1  & 3        & 12       & 40         & 21        & 0         \\
                2  & 5        & 4        & 18         & 35        & 14        \\
                3  & 1        & 13       & 30         & 30        & 2         \\
                4  & 8        & 6        & 23         & 28        & 11        \\
                5  & 6        & 7        & 20         & 25        & 18        \\
                6  & 1        & 5        & 17         & 29        & 24        \\
                7  & 5        & 4        & 20         & 35        & 12        \\
                8  & 5        & 3        & 25         & 31        & 12        \\
                9  & 8        & 8        & 23         & 26        & 11        \\
                10 & 4        & 11       & 29         & 29        & 3         \\
                11 & 2        & 4        & 20         & 34        & 16        \\
                12 & 17       & 17       & 26         & 10        & 6         \\
                13 & 1        & 11       & 31         & 24        & 9         \\
                14 & 6        & 22       & 23         & 22        & 3         \\
                15 & 3        & 5        & 12         & 31        & 25        \\
                16 & 4        & 11       & 27         & 30        & 4         \\
                17 & 3        & 11       & 18         & 31        & 13        \\
                \bottomrule
            \end{tabularx}
        \end{center}
        \caption{\label{tab:rawratingquestionresultstable} Raw results for the rating questions.}
    \end{table}

    \begin{table}[!htbp]
        \begin{center}
            \begin{tabularx}{\textwidth}{lXXXll}
                \toprule
                Id & Mean & Negative & Positive & Standard Deviation & Confidence Interval \\
                \midrule
                1  & 3.04 & 20.0\%   & 28.0\%   & 0.774              & \textpm 0.174       \\
                2  & 3.64 & 12.0\%   & 65.3\%   & 1.055              & \textpm 0.237       \\
                3  & 3.25 & 18.7\%   & 42.7\%   & 0.819              & \textpm 0.184       \\
                4  & 3.37 & 18.7\%   & 52.0\%   & 1.153              & \textpm 0.259       \\
                5  & 3.55 & 17.3\%   & 57.3\%   & 1.182              & \textpm 0.266       \\
                6  & 3.92 & 8.0\%    & 70.7\%   & 0.963              & \textpm 0.217       \\
                7  & 3.59 & 10.7\%   & 62.7\%   & 1.035              & \textpm 0.233       \\
                8  & 3.55 & 21.3\%   & 57.3\%   & 1.025              & \textpm 0.230       \\
                9  & 3.32 & 20.0\%   & 49.3\%   & 1.169              & \textpm 0.263       \\
                10 & 3.21 & 20.0\%   & 42.7\%   & 0.929              & \textpm 0.209       \\
                11 & 3.76 & 8.0\%    & 66.7\%   & 0.936              & \textpm 0.211       \\
                12 & 2.62 & 45.3\%   & 21.3\%   & 1.200              & \textpm 0.270       \\
                13 & 3.38 & 16.0\%   & 44.0\%   & 0.923              & \textpm 0.208       \\
                14 & 2.92 & 37.3\%   & 33.3\%   & 1.030              & \textpm 0.232       \\
                15 & 3.92 & 10.7\%   & 74.7\%   & 1.055              & \textpm 0.237       \\
                16 & 3.25 & 20.0\%   & 45.3\%   & 0.954              & \textpm 0.214       \\
                17 & 3.53 & 18.7\%   & 58.7\%   & 1.064              & \textpm 0.239       \\
                \bottomrule
            \end{tabularx}
        \end{center}
        \caption{\label{tab:ratingquestionresultstable} Calculated results for the rating questions.}
    \end{table}

\end{document}